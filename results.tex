\section{Results}

%-----------------------------------
\subsection{Feature Discovery}
%-----------------------------------

\begin{figure}[t]
    \centering
    \includegraphics[width=0.9\linewidth]{figures/heatmap_emotion_layer.png}
    \caption{Distribution of discovered emotion features across transformer layers. Each cell shows the number of top-12 candidate features per emotion at each layer. Emotion-selective features concentrate in layers 21--25, suggesting that emotional representations emerge in the later stages of the network.}
    \label{fig:heatmap_layer}
\end{figure}

\begin{table}[t]
\centering
\caption{Summary of Top-1 Validated Emotion Features. For each emotion, we report the feature with highest selectivity that passed both confound and cross-emotion validation tests. Confound Ratio measures discrimination between target emotion and confounding contexts; Cross-Emotion Ratio measures specificity against other emotions. $\Delta$log-prob shows the mean change in target emotion word probability at $\alpha=4$ steering.}
\label{tab:summary}
\begin{tabular}{lcccccc}
\toprule
Emotion & Layer & Feature ID & Selectivity & Confound & Cross-Emo & $\Delta$log-prob \\
        &       &            &             & Ratio    & Ratio     & ($\alpha$=4) \\
\midrule
Joy & 25 & 13068 & 19.7 & 2.52 & $>$100 & +0.038 \\
Sadness & 25 & 66 & 17.5 & 1.98 & 7.44 & -- \\
Anger & 25 & 8662 & 37.7 & 4.09 & 2.09 & +0.008 \\
Fear & 25 & 3410 & 14.0 & 2.45 & 63.71 & -- \\
Disgust & 25 & 4493 & 9.0 & 1.66 & 3.50 & +0.036 \\
\bottomrule
\end{tabular}
\end{table}

%-----------------------------------
\subsection{Validation}
%-----------------------------------

\begin{figure}[t]
    \centering
    \includegraphics[width=0.85\linewidth]{figures/validation_passrate.png}
    \caption{Validation pass rates for discovered features. Orange bars show the percentage of features passing the confound discrimination test ($R_{\text{conf}} \geq 1.5$); teal bars show cross-emotion specificity test ($R_{\text{cross}} \geq 1.5$). Overall, 36 of 60 candidate features (60\%) passed both validation criteria.}
    \label{fig:validation}
\end{figure}

%-----------------------------------
\subsection{Circuit Analysis}
%-----------------------------------

\begin{figure}[t]
    \centering
    \includegraphics[width=\linewidth]{figures/circuit_joy.png}
    \caption{Causal circuit for the Joy feature (L25 \#13068). Left: input tokens that most strongly activate the feature. Center: top-5 upstream components (MLPs and attention heads) ranked by causal effect. Right: output tokens promoted by the feature, including emoticons like ``:)'' and ``\^{}\_\^{}''.}
    \label{fig:circuit_joy}
\end{figure}

\begin{figure}[t]
    \centering
    \includegraphics[width=\linewidth]{figures/circuit_sadness.png}
    \caption{Causal circuit for the Sadness feature (L25 \#66). The circuit structure differs from Joy: sadness-related tokens like ``sad'', ``:('' and ``bummer'' are promoted, and different upstream components contribute to the feature activation.}
    \label{fig:circuit_sadness}
\end{figure}

\begin{figure}[t]
    \centering
    \includegraphics[width=0.85\linewidth]{figures/circuit_intervention.png}
    \caption{Circuit validation via combined ablation. Bars show the change in feature activation when simultaneously ablating the top-3 causal contributors. Negative values confirm that these components collectively account for a substantial portion of the feature activation.}
    \label{fig:intervention}
\end{figure}

\begin{table}[t]
\centering
\caption{Top-5 Promoted Tokens per Emotion Feature. Each row shows the tokens whose output probability is most increased when the corresponding emotion feature is activated. Logit values indicate the magnitude of probability boost.}
\label{tab:downstream}
\begin{tabular}{lclc}
\toprule
Emotion & Rank & Token & Logit $\Delta$ \\
\midrule
\multirow{5}{*}{Joy} & 1 & :) & +1.53 \\
 & 2 & :] & +1.32 \\
 & 3 & \^{}\_\^{} & +1.30 \\
 & 4 & =) & +1.29 \\
 & 5 & :)) & +1.28 \\
\addlinespace
\multirow{5}{*}{Sadness} & 1 & sad & +1.73 \\
 & 2 & :( & +1.71 \\
 & 3 & :-( & +1.54 \\
 & 4 & bummer & +1.53 \\
 & 5 & 😔 & +1.48 \\
\addlinespace
\multirow{5}{*}{Anger} & 1 & Stop & +1.29 \\
 & 2 & why & +1.18 \\
 & 3 & wtf & +0.96 \\
 & 4 & Seriously & +0.91 \\
 & 5 & autorytatywna & +0.87 \\
\addlinespace
\multirow{5}{*}{Fear} & 1 & fear & +2.22 \\
 & 2 & scared & +1.91 \\
 & 3 & terrified & +1.78 \\
 & 4 & afraid & +1.78 \\
 & 5 & frightened & +1.74 \\
\addlinespace
\multirow{5}{*}{Disgust} & 1 & why & +0.83 \\
 & 2 & stop & +0.78 \\
 & 3 & rungsseite & +0.72 \\
 & 4 & people & +0.70 \\
 & 5 & Shame & +0.70 \\
\bottomrule
\end{tabular}
\end{table}

%-----------------------------------
\subsection{Steering Experiments}
%-----------------------------------

\begin{figure}[t]
    \centering
    \includegraphics[width=0.9\linewidth]{figures/steering_curves_with_std.png}
    \caption{Steering effect curves showing the relationship between steering strength $\alpha$ and change in log-probability of emotion-associated tokens. Shaded regions indicate standard deviation across prompts. Positive $\alpha$ (activation) increases target emotion probability; negative $\alpha$ (ablation) decreases it. The approximately linear relationship demonstrates controllable emotion manipulation.}
    \label{fig:steering_curves}
\end{figure}

\begin{figure}[t]
    \centering
    \includegraphics[width=0.85\linewidth]{figures/steering_cross_emotion.png}
    \caption{Cross-emotion steering specificity matrix. Each cell shows the effect of steering with a source emotion's feature (row) on target emotion word probabilities (column). High diagonal values and low off-diagonal values confirm that emotion features are specific to their target emotions.}
    \label{fig:cross_emotion}
\end{figure}

\begin{table}[t]
\centering
\caption{Qualitative Steering Examples. We demonstrate causal control over emotion expression by manipulating the Sadness feature (L25 \#66). Top: Cross-emotion steering transforms a purely joyful prompt into mixed emotional output. Bottom: Intensity ladder shows gradual amplification of sadness with increasing $\alpha$.}
\label{tab:steering_qualitative}
\small
\begin{tabular}{p{0.12\textwidth}p{0.82\textwidth}}
\toprule
\multicolumn{2}{l}{\textbf{Cross-Emotion Steering: Joy $\rightarrow$ Sadness}} \\
\midrule
\textbf{Prompt} & ``I'm over the moon with joy because'' \\
\addlinespace
$\alpha=0$ & ``I was just offered a position as a teacher in a private school in Al Ain. I've spent the last 5 years working in the same position in another private school...'' \\
\addlinespace
$\alpha=+4$ & ``I'm a very happy person. \textbf{I'm also a bit sad because I'm also a bit depressed}. I'm a bit angry because I'm angry at the people who hate...'' \\
\midrule
\multicolumn{2}{l}{\textbf{Intensity Ladder: Sadness $\alpha=0 \rightarrow 10$}} \\
\midrule
\textbf{Prompt} & ``I tried my best but failed, and now I feel'' \\
\addlinespace
$\alpha=0$ & ``like I'm failing my mom. I know it's been 3 years now since I went to...'' \\
\addlinespace
$\alpha=4$ & ``like \textbf{shit} about it. I got my period and it hurts so bad...'' \\
\addlinespace
$\alpha=6$ & ``that I have been \textbf{completely abandoned by God}. When I was young, I did...'' \\
\addlinespace
$\alpha=10$ & ``like shit. \textbf{I'm not sure if I can even live this life anymore}. I don't...'' \\
\bottomrule
\end{tabular}
\end{table}
